% Options for packages loaded elsewhere
\PassOptionsToPackage{unicode}{hyperref}
\PassOptionsToPackage{hyphens}{url}
\PassOptionsToPackage{dvipsnames,svgnames,x11names}{xcolor}
%
\documentclass[
  12pt,
]{article}
\usepackage{amsmath,amssymb}
\usepackage{lmodern}
\usepackage{iftex}
\ifPDFTeX
  \usepackage[T1]{fontenc}
  \usepackage[utf8]{inputenc}
  \usepackage{textcomp} % provide euro and other symbols
\else % if luatex or xetex
  \usepackage{unicode-math}
  \defaultfontfeatures{Scale=MatchLowercase}
  \defaultfontfeatures[\rmfamily]{Ligatures=TeX,Scale=1}
\fi
% Use upquote if available, for straight quotes in verbatim environments
\IfFileExists{upquote.sty}{\usepackage{upquote}}{}
\IfFileExists{microtype.sty}{% use microtype if available
  \usepackage[]{microtype}
  \UseMicrotypeSet[protrusion]{basicmath} % disable protrusion for tt fonts
}{}
\makeatletter
\@ifundefined{KOMAClassName}{% if non-KOMA class
  \IfFileExists{parskip.sty}{%
    \usepackage{parskip}
  }{% else
    \setlength{\parindent}{0pt}
    \setlength{\parskip}{6pt plus 2pt minus 1pt}}
}{% if KOMA class
  \KOMAoptions{parskip=half}}
\makeatother
\usepackage{xcolor}
\usepackage[margin=1in]{geometry}
\usepackage{longtable,booktabs,array}
\usepackage{calc} % for calculating minipage widths
% Correct order of tables after \paragraph or \subparagraph
\usepackage{etoolbox}
\makeatletter
\patchcmd\longtable{\par}{\if@noskipsec\mbox{}\fi\par}{}{}
\makeatother
% Allow footnotes in longtable head/foot
\IfFileExists{footnotehyper.sty}{\usepackage{footnotehyper}}{\usepackage{footnote}}
\makesavenoteenv{longtable}
\usepackage{graphicx}
\makeatletter
\def\maxwidth{\ifdim\Gin@nat@width>\linewidth\linewidth\else\Gin@nat@width\fi}
\def\maxheight{\ifdim\Gin@nat@height>\textheight\textheight\else\Gin@nat@height\fi}
\makeatother
% Scale images if necessary, so that they will not overflow the page
% margins by default, and it is still possible to overwrite the defaults
% using explicit options in \includegraphics[width, height, ...]{}
\setkeys{Gin}{width=\maxwidth,height=\maxheight,keepaspectratio}
% Set default figure placement to htbp
\makeatletter
\def\fps@figure{htbp}
\makeatother
\setlength{\emergencystretch}{3em} % prevent overfull lines
\providecommand{\tightlist}{%
  \setlength{\itemsep}{0pt}\setlength{\parskip}{0pt}}
\setcounter{secnumdepth}{5}
\newlength{\cslhangindent}
\setlength{\cslhangindent}{1.5em}
\newlength{\csllabelwidth}
\setlength{\csllabelwidth}{3em}
\newlength{\cslentryspacingunit} % times entry-spacing
\setlength{\cslentryspacingunit}{\parskip}
\newenvironment{CSLReferences}[2] % #1 hanging-ident, #2 entry spacing
 {% don't indent paragraphs
  \setlength{\parindent}{0pt}
  % turn on hanging indent if param 1 is 1
  \ifodd #1
  \let\oldpar\par
  \def\par{\hangindent=\cslhangindent\oldpar}
  \fi
  % set entry spacing
  \setlength{\parskip}{#2\cslentryspacingunit}
 }%
 {}
\usepackage{calc}
\newcommand{\CSLBlock}[1]{#1\hfill\break}
\newcommand{\CSLLeftMargin}[1]{\parbox[t]{\csllabelwidth}{#1}}
\newcommand{\CSLRightInline}[1]{\parbox[t]{\linewidth - \csllabelwidth}{#1}\break}
\newcommand{\CSLIndent}[1]{\hspace{\cslhangindent}#1}
\usepackage{polyglossia}
\setmainlanguage{turkish}
\usepackage{booktabs}
\usepackage{caption}
\captionsetup[table]{skip=10pt}
\ifLuaTeX
  \usepackage{selnolig}  % disable illegal ligatures
\fi
\IfFileExists{bookmark.sty}{\usepackage{bookmark}}{\usepackage{hyperref}}
\IfFileExists{xurl.sty}{\usepackage{xurl}}{} % add URL line breaks if available
\urlstyle{same} % disable monospaced font for URLs
\hypersetup{
  pdftitle={Covid-19 Pandemisinin Toplumun Mutluluğu Üzerindeki Etkisi},
  pdfauthor={Ömer Yiğit Kaya{[}\^{}1{]}},
  colorlinks=true,
  linkcolor={Maroon},
  filecolor={Maroon},
  citecolor={Blue},
  urlcolor={blue},
  pdfcreator={LaTeX via pandoc}}

\title{Covid-19 Pandemisinin Toplumun Mutluluğu Üzerindeki Etkisi}
\author{Ömer Yiğit Kaya{[}\^{}1{]}}
\date{}

\begin{document}
\maketitle

\hypertarget{giriux15f}{%
\section{Giriş}\label{giriux15f}}

Covid-19, 2019 yılı sonunda Çin'in Wuhan kentinde ortaya çıkan ve kısa sürede tüm dünyaya yayılan yeni tip bir koronavirüs hastalığıdır. Dünya Sağlık Örgütü tarafından pandemi ilan edilen Covid-19, insan sağlığı, ekonomi, eğitim, sosyal hayat gibi pek çok alanda ciddi sorunlara neden olmuştur. Covid-19 salgını, bireylerin yaşam koşullarını, ruh sağlığını ve mutluluk düzeylerini de etkilemiştir.

(\protect\hyperlink{ref-bakkeli2021health}{Bakkeli, 2021})'nin araştırmış olduğu \textbf{Norveç'te Covid-19 salgını öncesinde ve sırasında mutluluk düzeyi}isimli çalışmayı Türkiye'ye uyarlamak istedim.Bunun için başlangıçta {[}Karataş (\protect\hyperlink{ref-karatacs2020covid}{2020}){]}(\protect\hyperlink{ref-gungorer2020covid}{Güngörer, 2020})(\protect\hyperlink{ref-ustun2020covid}{Üstün ve Özçiftçi, 2020})'nin hazırlamış olduğu makaleleri okuyarak bilgi edindim daha sonrasında (\protect\hyperlink{ref-raabe2020satisfaction}{Raabe vd., 2020})detay bilgi ve bu konu hakkında doğru veriyi nereden toplayabilirim onu kavramaya çalıştım.

Veri setini, genel mutluluk düzeyi, kişisel sağlıktan duyulan memnuniyet ve sosyal hayattan duyulan memnuniyet olacak şekilde, TÜİK'e ait \textbf{Merkezi Dağıtım Sisteminden} edindim.Veri setinde cinsiyet, memnuniyet düzeyi (çok mutlu, mutlu, orta, mutsuz, çok mutsuz) ve verinin toplandığı yıl(2018,2019,2020,2021,2022) gibi gözlemler bulunmaktadır.

Covid-19 salgını ve onun yanında getirdiği tüm etkilerin, toplumun mutluluğu üzerindeki etkilerini incelemek için veri setinde yer alan 2018 ve 2019 verilerini Covid-19 öncesi, 2020 ve 2021 verilerini Covid-19 sırası, 2022 verisini ise Covid-19 sonrası olarak inceleyeceğim.

\hypertarget{uxe7alux131ux15fmanux131n-amacux131}{%
\subsection{Çalışmanın Amacı}\label{uxe7alux131ux15fmanux131n-amacux131}}

Covid-19 salgını, tüm dünyada insanların yaşam koşullarını, ruh sağlığını ve mutluluk düzeylerini etkileyen önemli bir olaydır. Salgın nedeniyle bireylerin karşılaştıkları korku, belirsizlik, kısıtlama, yalnızlık, işsizlik, yoksulluk gibi olumsuz durumlar, depresyon, anksiyete, stres gibi psikolojik sorunlara yol açabilir. Bu da bireylerin yaşam doyumunu ve mutluluklarını azaltabilir. Bu nedenle salgının bireylerin mutluluk düzeyi üzerindeki etkisini araştırmak hem akademik hem de sosyal açıdan önemlidir.

Bu çalışmanın amacı, Covid-19 salgınının Türkiye'deki bireylerin mutluluk düzeyi üzerindeki etkisini istatiksel olarak incelemektir. Bu amaçla bireylerin mutluluk düzeyleri ile kişisel sağlıktan duyulan memnuniyet ve sosyal hayattan duyulan memnuniyet arasındaki ilişki analiz edilecektir.

Bu çalışmanın önemi, Covid-19 salgınının bireylerin mutluluk düzeyi üzerindeki etkisini ortaya koymasıdır.

\hypertarget{literatuxfcr}{%
\subsection{Literatür}\label{literatuxfcr}}

Bu bölümde konu ile ilgili olarak okuduğunuz makaleleri referans vererek tartışınız. \textbf{Her makaleyi ayrı başlık altında tek tek özetlemeyiniz.} Literatür taramasında \textbf{en az dört} makaleye atıf yapılması ve bu makalelerden \textbf{en az ikisinin İngilizce} olması gerekmektedir.

\newpage

\hypertarget{references}{%
\section{Kaynakça}\label{references}}

\hypertarget{refs}{}
\begin{CSLReferences}{1}{0}
\leavevmode\vadjust pre{\hypertarget{ref-bakkeli2021health}{}}%
Bakkeli, N. Z. (2021). Health, work, and contributing factors on life satisfaction: A study in Norway before and during the COVID-19 pandemic. \emph{SSM-population Health}, \emph{14}, 100804.

\leavevmode\vadjust pre{\hypertarget{ref-gungorer2020covid}{}}%
Güngörer, F. (2020). Covid-19'un toplumsal kurumlara etkisi. \emph{Y{ü}z{ü}nc{ü} Y{ı}l {Ü}niversitesi Sosyal Bilimler Enstit{ü}s{ü} Dergisi}, (Salg{ı}n Hastal{ı}klar {Ö}zel Say{ı}s{ı}), 393-328.

\leavevmode\vadjust pre{\hypertarget{ref-karatacs2020covid}{}}%
Karataş, Z. (2020). COVID-19 pandemisinin toplumsal etkileri, de{ğ}i{ş}im ve g{ü}{ç}lenme. \emph{T{ü}rkiye Sosyal Hizmet Ara{ş}t{ı}rmalar{ı} Dergisi}, \emph{4}(1), 3-17.

\leavevmode\vadjust pre{\hypertarget{ref-raabe2020satisfaction}{}}%
Raabe, I. J., Ehlert, A., Johann, D. ve Rauhut, H. (2020). Satisfaction of scientists during the COVID-19 pandemic lockdown. \emph{Humanities and Social Sciences Communications}, \emph{7}(1), 1-7.

\leavevmode\vadjust pre{\hypertarget{ref-ustun2020covid}{}}%
Üstün, Ç. ve Özçiftçi, S. (2020). COVID-19 pandemisinin sosyal ya{ş}am ve etik d{ü}zlem {ü}zerine etkileri: Bir de{ğ}erlendirme {ç}al{ı}{ş}mas{ı}. \emph{Anatolian Clinic the Journal of Medical Sciences}, \emph{25}(Special Issue on COVID 19), 142-153.

\end{CSLReferences}

\end{document}
